\chapter{Linear algebra}
\label{chap:Linear algebra}

When it comes to linear algebra, don't try to write your own routines, always use the BLAS (Basic Linear Algebra Subprograms) and \indexentry{LAPACK} (Linear Algebra PACKage)\footnote{\url{http://www.netlib.org/lapack/}.
Under Ubuntu, install\index{LAPACK!installation} the \texttt{liblapack-dev} package to obtain BLAS and \indexentry{LAPACK}, and don't forget to add \texttt{-lblas}\index{compiler flags!-lblas} and \texttt{-llapack}\index{compiler flags!-llapack} to \texttt{\$(LIBS)} in your \texttt{Makefile}.} libraries.
Everyone does.
Maple, Mathematica, MATLAB and Python (with NumPy and SciPy) all use it under the hood.
The \texttt{dot\_product} and \texttt{matmul} intrinsics in Fortran are implemented with it.
It's installed on every supercomputer and contains some of the most sophisticated and optimized code known to man.
The fact that these libraries are written in Fortran is actually one of the main reasons to use Fortran for high-performance computing.
Nevertheless, BLAS and \indexentry{LAPACK} can be a bit low-level sometimes, and the Fortran~77 syntax doesn't help matters much.
For your convenience, we therefore provide two functions for common tasks: calculating the inverse and the eigenvalues + eigenvectors of a real symmetric matrix.
This should help you get started.
For other functions, use Google.
BLAS and \indexentry{LAPACK} are quite extensively documented.
\lstinputlisting[float=htbp,label=lst:invertMatrix]{examples/invertMatrix.f90}

\lstinputlisting[float=htbp,label=lst:calculateEigensystem]{examples/calculateEigensystem.f90}

