\chapter{Coding style}
\label{chap:Coding style}

\begin{quote}\small
  \emph{``Consistently separating words by spaces became a general custom about the tenth century A.D., and lasted until about 1957, when FORTRAN abandoned the practice.''} \\ \hspace*{\fill}---Sun FORTRAN Reference Manual
\end{quote}

\begin{quote}\small
  \emph{``Simplicity is prerequisite for reliability.''} \\ \hspace*{\fill}---Edsger Dijkstra
\end{quote}
A major part of writing clean code is following a sensible coding style.
Although style is subjective and a matter of taste, consistency is not.
Which style you pick is not as important as sticking to it.
However, certain styles are objectively better than others.
While we won't force a particular style, you will make it a lot easier for your instructors to read your code if you follow these guidelines, and that's always a good thing.
Remember: you will be spending a lot more time reading your code than writing it; keep it clean.
\begin{itemize}
  \item\textbf{Line width}---Limit the line width of your source to 80 characters.
    This has been standard practice for decades, and vastly improves both readability and the ability to cleanly print your code.
    If you do need more than 80 characters on a line, use the Fortran continuation character `\texttt{\&}'.
  \item\textbf{Indentation}---Indent every block, be it  a \keyword{program}, \keyword{module}, \linebreak \keyword{subroutine}, \keyword{function}, \keyword{do} or \keyword{if} block by 2 or 4 spaces.
    \emph{Don't use tabs!} Code that looks properly aligned on your machine may look horrible on another because of a different tab setting.
    Proper indentation makes it clear where control structures begin and end, and is a major boon to readability.
    If the indentation puts you over the 80 character limit, that's an indication that you may need to refactor your code intro smaller subroutines.
    Four levels of indentation should generally be the maximum.
  \item\textbf{Spaces}---Put spaces around binary operators, including \texttt{::}, and after keywords and commas, not after unary operators, function names or around parentheses.
\begin{lstlisting}[caption=,nolol,float=]
  real(8) :: x, y
  x = func1()
  if ((x < 0) .or. (x > 1)) then
      y = func2(-2, 2*x + 3)
  end if
\end{lstlisting}
  \item\textbf{Naming}---Identifiers in Fortran can't contain spaces.
    Use \linebreak \texttt{CamelCase} to separate words in program and module names, and \linebreak \texttt{lower\_case\_with\_underscores} for function and variable names.
    Names should be descriptive, but they don't have to be overly long.
    You can use \texttt{calc\_deriv} instead of \texttt{calculate\_derivative}, but naming your function \texttt{cd} is a shooting offense.
    Unfortunately, until the appearance of Fortran 90, only the first six characters of an identifier were significant, leading to such beautifully transparent function names as \texttt{DSYTRF}\footnote{DSYTRF computes the factorization of a real symmetric matrix using the Bunch-Kaufman diagonal pivoting method, in case you wondered.} in legacy codes like \idx{LAPACK}.
    They could get away with it because it was the eighties; you can't.
    One exception to this rule is using short identifiers like \texttt{i}, \texttt{j} and \texttt{k} as loop counters, or \texttt{x} and \texttt{y} as arguments to mathematical functions.
    However, these should always be local variables or function arguments, never global identifiers.
  \item\textbf{Functions and subroutines}---Functions and subroutines should follow the Unix philosophy: do one thing and do it well.
    Split your program into small self-contained subunits.
    This makes it a lot easier to understand your code, and to modify it afterwards.
    If a function does not fit in its entirety on a single screen, it is too long---split it up.
    Don't worry about the overhead of calling a function.
    If you use optimization flags like \texttt{-O3}, the compiler will try to inline them anyway.
  \item\textbf{Modules}---Separate your program into logical units.
    Combine related functions and subroutines into modules\footnote{It's also convenient to put all physical parameters (number of particles, $\pi$, $c$, $k_B$, etc.) into their own module, even if they're just constant numbers. You can then ``load'' the parameters into your subroutines with the \keyword{use} statement instead of having absurdly long lists of arguments, or worse, global variables.}.
    For example, a single module with function integrators, or ODE solvers. 
    If you use the object-oriented features of Fortran, it is also good practice to create separate modules for classes, \eg, a sparse-matrix object.
  \item\textbf{Comments}---Comments are kind of a double-edged sword.
    If used wrongly, they can actually hurt readability.
    The golden rule is: never explain \emph{how} it works, just \emph{what} it does.
    The how should be evident from the code itself.
    \emph{``Good code is its own best documentation.''} If you follow the other guidelines regarding naming and keeping functions small, you can generally suffice with a small comment at the beginning of every function or subroutine.
\end{itemize}
One final point, not of style, but important nonetheless: start every program and module with \keyword{implicit none}\marginnote{The \texttt{-fimplicit-none}\index{compiler flags!-fimplicit-none} compiler flag will also turn off implicit typing globally.}  \index{Fortran!implicit none@\keyword{implicit none}}.
This is an unfortunate holdover from older versions of Fortran; it forces you to explicitly declare all variables, instead of allowing Fortran to infer the type from the name.
Forgetting \keyword{implicit none} and making a typo in one of your variable names is a major source of bugs.
\emph{Never ever omit this!}
