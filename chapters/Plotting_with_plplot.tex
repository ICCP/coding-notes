\chapter{Plotting with PLplot}
\label{chap:PLplot}

Plotting the results of your calculation from Fortran is easiest with \href{http://plplot.sourceforge.net/}{PLplot}.
PLplot is a cross-platform scientific plotting package with support for multiple programming languages.
Under Ubuntu, you need to install the packages \texttt{libplplot-dev} and \texttt{plplot11-driver-cairo}.
The \texttt{plplot-doc} package might also come in handy.
If you have it installed, you can access the documentation of, \eg, the \texttt{plline} subroutine by typing
\begin{verbatim}
$ man plline
\end{verbatim}
on the command line.

In order to successfully compile a program using PLplot, you need quite a few compiler flags and libraries.
However, Ubuntu (and most other distributions) is smart enough to figure that out for you.
Add the following two lines near the top of your \texttt{Makefile}, but after the declarations of \texttt{FFLAGS} and \texttt{LIBS}:
\begin{verbatim}
FFLAGS += $(shell pkg-config --cflags plplotd-f95)
LIBS += $(shell pkg-config --libs plplotd-f95)
\end{verbatim}
The \texttt{pkg-config} command will generate the correct compiler flags on the fly.

\section{Plotting functions}

As a quick introduction to using PLplot, we'll go through \autoref{lst:PLplotDemo} step by step:
\lstinputlisting[label=lst:PLplotDemo]{examples/PLplotDemo.f90}
%\lstinputlisting[caption=PLplotDemo (continued),float=htbp,label=lst:PLplotDemo2,firstline=44,firstnumber=44]{examples/PLplotDemo.f90}
\begin{itemize}
  \item \texttt{\keyword{use} plplot} imports the PLplot functions and subroutines into our program.
  \item \texttt{\keyword{call} plparseopts(PL\_PARSE\_FULL)} parses any command-line arguments of your program that are relevant for PLplot (see below).
  \item By default, PLplot plots all figures on a black background with red axes.
    This is inconvenient, especially for printed figures.
    Lines 11--19 change the color scheme of color map 0 (which is used for line plots) to that of gnuplot (with the exception of color 0, since that's the background color).
  \item All plotting commands must be given between the calls to \texttt{plinit} and \texttt{plend}.
  \item \texttt{plot\_x2} plots the function $y=x^2$ for \texttt{n} values between \texttt{x1} and \texttt{x2}.
  \item The \texttt{linspace} subroutine generates a linearly spaced array, similar to the MATLAB function of the same name.
  \item \texttt{\keyword{call} plcol0(7)} sets the current foreground color to 7 (black), which we use for the axes and labels.
    The call to \texttt{plenv} sets the minimum and maximum values of the axes, scales the plot to fill the screen (the first \texttt{0} parameter) and draws a box with tick marks (the second \texttt{0}).
    \texttt{pllab} is then used to set the $x$ and $y$ labels and the title of the plot.
  \item Finally, we draw a red line (color 1) through the points with \texttt{plline} and draw the points themselves in green (color 2) with \texttt{plpoin}.
    The points are drawn with the `+' glyph (code 2).
\end{itemize}
If the program compiles successfully, you can run it with
\begin{verbatim}
$ ./plplotdemo -dev xcairo
\end{verbatim}
The argument \texttt{-dev xcairo} tells PLplot to use the Cairo X Windows driver, which will open a window showing the plot.
It is also possible to generate a PostScript image by typing
\begin{verbatim}
$ ./plplotdemo -dev pscairo -o demo.ps
\end{verbatim}
and similarly a PDF with \texttt{pdfcairo}. You can find out about the command-line parameters by running
\begin{verbatim}
$ ./plplotdemo -h
\end{verbatim}
or by specifying no arguments at all.

\section{3D animations}

For some programs, such as a molecular dynamics simulation, it is convenient to be able to watch the motion of particles in real time.
For this we'll make use of the 3D plotting facilities of PLplot.
First, we setup the viewport with
\lstinputlisting[lastline=10]{examples/PLplot3D.f90} % Note, these 3 listings all draw from the same source file. Any changes there require a MANUAL reindexing of the firstline and lastline values!
Since the animation will play real-time, we specify the Cairo X Windows driver directly with \texttt{plsdev} instead of parsing the command line.
Lines 5--7 map the world coordinates---bounded by \texttt{xmin}, \texttt{xmax}, \emph{etc.}---to the screen.
This will result in a box which is rotated by 45 degrees around the $z$ and $x$ axes.
We end the program with
\lstinputlisting[firstline=11, lastline=12, firstnumber=11, caption=, nolol]{examples/PLplot3D.f90}
The call to \texttt{plspause} makes sure the program terminates after the animation finishes, instead of leaving the last frame on screen indefinitely.

We plot particles by calling the following subroutine:
\lstinputlisting[firstline=14, firstnumber=14, caption=, nolol]{examples/PLplot3D.f90}
\texttt{xyz} is a $3\times n$ array, with the first row containing the $x$ coordinates of the particles, the second the $y$, and the third the $z$ coordinates.
We first clear the screen with \texttt{plclear}, then draw the axis with \texttt{plbox3}, and finally the particles with \texttt{plpoin3}.
The particles are drawn as small circles (code 4).
Finally, we update the screen by calling \texttt{plflush}.
Calling \texttt{plot\_points} repeatedly results in a smooth animation.
Although this works quite well for checking the movement of particles, PLplot is not a 3D engine and won't run at more than about 100 frames per second.
We therefore recommend not plotting every step in your simulation, but skipping a few frames.
For more documentation on PLplot, Google is your friend.

