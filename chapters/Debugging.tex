\Chapter{Debugging}{(or, where you'll spend 90\% of your time)}
\label{chap:Debugging}

\begin{quote}\small
  \emph{``Everyone knows that debugging is twice as hard as writing a program in the first place.
  So if you're as clever as you can be when you write it, how will you ever debug it?'' \\
  ``The most effective debugging tool is still careful thought, coupled with judiciously placed print statements.''} \\\hspace*{\fill}---Brian Kernighan
\end{quote}
The \texttt{MyProg} program does not contain any bugs (fingers crossed) and works as expected.
However, any non-trivial program will be buggy after your first attempt at writing it.
\emph{In this course you will spend the majority of your time debugging your algorithms and your code, not on the actual writing!} To ease your pain, we recommend following the golden rules of debugging:
\begin{itemize}
  \item\textbf{Always fix the first error first.} Errors have a tendency to cascade, with a simple typo early on generating loads of errors in the rest of your program.
    If you find yourself with a ton of unmanageable issues, use the \texttt{-Wfatal-errors} flag\index{compiler flags!-Wfatal-errors} to halt compilation after the first error so you can fix them individually.
  \item\textbf{Don't ignore warnings.} Warnings are often an indication that your code doesn't actually do what you think it does.
    And if it does happen to work, it's an indication of bad programming practice, or worse, style.
    You can use the \texttt{-Werror} flag\index{compiler flags!-Werror} to elevate warnings into errors, forcing you to fix them before the compiler will produce an executable.
  \item\textbf{Test early, test often.} No matter how large the project, always start with a trivial program and make sure that works.
    Then add functionality bit by bit, each time making sure it compiles and works.
  \item\textbf{Produce lots of output.} You might think that there can't possibly be a bug in that beautiful expression, but you're wrong.
    There are countless innocuous subtleties to programming that can result in inaccuracies or unintended side effects.
    Test your program by printing the intermediate results.
  \item\textbf{Keep a rubber ducky on your desk.} You've been staring at your code for hours, it's obviously correct, yet it doesn't work.
    In that case the best thing to do is to try and explain your code to someone else.
    It doesn't matter who it is; they don't have to understand Fortran, or even programming in general.
    It can be your partner, your mother, your dog, or even a rubber ducky.
    Just explain it line-by-line; ten to one says you'll find the bug.
\end{itemize}
Even if you do follow these rules---\emph{and you should!}---you will run into bugs that you don't understand.
Often they can be fixed by backtracking and checking every intermediate step, but bugs can be elusive.
In those cases a debugger comes in handy.
To use a debugger under Ubuntu, install the \texttt{gdb} and \texttt{ddd} packages.
You'll also need to compile your program with the \texttt{-g} flag\index{compiler flags!-g} to generate debug symbols, and turn off any optimizations as they might reorder expressions in your code, \eg,
\begin{verbatim}
$ gfortran -Wall -Wextra -march=native -g myprog.f90 -o myprog
\end{verbatim}
If the program compiles, you can invoke the debugger with
\begin{verbatim}
$ ddd ./myprog
\end{verbatim}
This will open a window showing the source code of your program and a GDB console (the actual debugger).
You can run the program by pressing the `Run' button, and interrupt a running program with `Interrupt'.
You can also insert so-called breakpoints into your program by right-clicking on a line and selecting `Set Breakpoint'.
If you then run your program, it will execute until it encounters the breakpoint, where it will pause.
You can then step through your code by clicking `Step' or `Next'.
`Step' will move to the next statement in your code.
`Next' does the same, but it will step over subroutines instead of entering them.
You can inspect variables by right-clicking on the name and selecting `Print x' or `Display x'.

Play around with the debugger for a while until you're comfortable using it.
It will come in very handy when you're hunting bugs, and as we said before, that's what you'll be spending most of your time on.
